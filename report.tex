\documentclass[12pt,a4paper]{article}
\usepackage{amsmath}
\usepackage{etex,datetime,setspace,latexsym,amssymb,amsmath,amsthm}
\usepackage{fancybox,dialogue,float,wrapfig,enumerate,microtype}
\usepackage{verbatim,xcolor,multicol,titlesec,tabularx,mdframed}
\usepackage{ circuitikz }


\usepackage[utf8]{inputenc}
\usepackage[pdftex]{hyperref}
\usepackage[margin=2cm,bottom=3cm,footskip=15mm]{geometry}
\parindent0cm
\parskip0.5em

\usepackage{tikz}
\usetikzlibrary{arrows,trees,positioning,shapes,patterns}
\usetikzlibrary{intersections,calc,fpu,decorations.pathreplacing}

\usepackage[T1]{fontenc} % better fonts

% Haskell code listings in our own style
\usepackage{listings,color}
\definecolor{lightgrey}{gray}{0.35}
\definecolor{darkgrey}{gray}{0.20}
\definecolor{lightestyellow}{rgb}{1,1,0.92}
\definecolor{dkgreen}{rgb}{0,.2,0}
\definecolor{dkblue}{rgb}{0,0,.2}
\definecolor{dkyellow}{cmyk}{0,0,.7,.5}
\definecolor{lightgrey}{gray}{0.4}
\definecolor{gray}{gray}{0.50}
\lstset{
  language        = Haskell,
  basicstyle      = \scriptsize\ttfamily,
  keywordstyle    = \color{dkblue},     stringstyle     = \color{red},
  identifierstyle = \color{dkgreen},    commentstyle    = \color{gray},
  showspaces      = false,              showstringspaces= false,
  rulecolor       = \color{gray},       showtabs        = false,
  tabsize         = 8,                  breaklines      = true,
  xleftmargin     = 8pt,                xrightmargin    = 8pt,
  frame           = single,             stepnumber      = 1,
  aboveskip       = 2pt plus 1pt,
  belowskip       = 8pt plus 3pt
}
\lstnewenvironment{code}[0]{}{}

% only shown, not compiled:
\lstnewenvironment{showCode}[0]{\lstset{numbers=none}}{}

% only compiled, not shown:
\newcommand{\hide}[1]{}

% will the real phi please stand up
\renewcommand{\phi}{\varphi}

\newcommand{\gor}{\mathbin{\rotatebox[origin=c]{-90}{$ \geqslant $}}}


% load hyperref as late as possible for compatibility
\usepackage[pdftex]{hyperref}
\hypersetup{
  colorlinks = true,
  pdfborder = {0 0 0},
  breaklinks = true,
  linktoc = all,
  linkcolor = blue,
  urlcolor = magenta
}
\pdfinfoomitdate=1
\pdftrailerid{}
\pdfsuppressptexinfo15


\title{My Report}
\author{Me}
\date{\today}
\hypersetup{pdfauthor={Me}, pdftitle={My Report}}

\begin{document}

\maketitle

\begin{abstract}
We implement an explicit model checker for bilateral state-based modal logic
(BSML), as described in \cite{Aloni2024}. BSML has been used to account for
Free-Choice (\textit{FC}) and related inferences which arise from speakers' distaste for
interpretations that verify a sentence by empty configuration  (\textit{neglect-zero tendency}).
\end{abstract}

Free-Choice (\textit{FC}) inferences are instances of conjunctive meaning being unexpectedly
derived from a disjunctive sentence. In the following example, a modalized disjunction yields
a conjunction of modals. 

\begin{enumerate}
\item You may go to the beach or to the cinema $\rightsquigarrow$ You may go to the beach
 and you may go to the cinema
\item $\Diamond(b\vee c)\rightsquigarrow\Diamond b \wedge \Diamond c$
\end{enumerate}


Such an inference is invalid in classical modal logic, since when $b$ is false and $c$ is true $\Diamond(b\vee c)$ is true but
$\Diamond b \wedge \Diamond c$ is false. Yet the inference in FC makes perfect sense in natural language. How
to account for the discrepancy then? Aloni (2022) explain FC inference as a consequence of the \textit{neglect-zero tendency},
the cognitive preference for representations of reality that verify a sentence by concrete occurrences rather
than an empty configuration. For example, though the sentence "I have no red hats" is verified both by a closet
of blue hats and by a closet with no hats, "blue hats" is less cognitively taxing than "no hats", the zero-model.

\tableofcontents

\clearpage

% We include one file for each section. The ones containing code should
% be called something.lhs and also mentioned in the .cabal file.


\section{How to use this?}

To generate the PDF, open \texttt{report.tex} in your favorite \LaTeX editor and compile.
Alternatively, you can manually do
\texttt{pdflatex report; bibtex report; pdflatex report; pdflatex report} in a terminal.

You should have stack installed (see \url{https://haskellstack.org/}) and
open a terminal in the same folder.

\begin{itemize}
  \item To compile everything: \verb|stack build|.
  \item To open ghci and play with your code: \verb|stack ghci|
  \item To run the executable from Section \ref{sec:Main}: \verb|stack build && stack exec myprogram|
  \item To run the tests from Section \ref{sec:simpletests}: \verb|stack clean && stack test --coverage|
\end{itemize}


\input{lib/Defs.lhs}

\input{exec/Main.lhs}

\input{test/simpletests.lhs}


\section{Conclusion}\label{sec:Conclusion}


\addcontentsline{toc}{section}{Bibliography}
\bibliographystyle{alpha}
\bibliography{references.bib}

\end{document}
