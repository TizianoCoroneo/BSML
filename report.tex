\documentclass[11pt,a4paper]{article}
\usepackage{amsmath, amsthm, mathtools, fontenc, graphicx}
\usepackage{etex,datetime,setspace,latexsym,amssymb,amsmath,amsthm}
\usepackage{fancybox,dialogue,float,wrapfig,enumerate,microtype}
\usepackage{verbatim,xcolor,multicol,titlesec,tabularx,mdframed}
\usepackage{ circuitikz }


\usepackage[utf8]{inputenc}
\usepackage[pdftex]{hyperref}
\usepackage[margin=2cm,bottom=3cm,footskip=15mm]{geometry}
\parindent0cm
\parskip0.5em

\usepackage{tikz}
\usetikzlibrary{arrows,trees,positioning,shapes,patterns}
\usetikzlibrary{intersections,calc,fpu,decorations.pathreplacing}

\usepackage[T1]{fontenc} % better fonts

% Haskell code listings in our own style
\usepackage{listings,color}
\definecolor{lightgrey}{gray}{0.35}
\definecolor{darkgrey}{gray}{0.20}
\definecolor{lightestyellow}{rgb}{1,1,0.92}
\definecolor{dkgreen}{rgb}{0,.2,0}
\definecolor{dkblue}{rgb}{0,0,.2}
\definecolor{dkyellow}{cmyk}{0,0,.7,.5}
\definecolor{lightgrey}{gray}{0.4}
\definecolor{gray}{gray}{0.50}
\lstset{
  language        = Haskell,
  basicstyle      = \scriptsize\ttfamily,
  keywordstyle    = \color{dkblue},     stringstyle     = \color{red},
  identifierstyle = \color{dkgreen},    commentstyle    = \color{gray},
  showspaces      = false,              showstringspaces= false,
  rulecolor       = \color{gray},       showtabs        = false,
  tabsize         = 8,                  breaklines      = true,
  xleftmargin     = 8pt,                xrightmargin    = 8pt,
  frame           = single,             stepnumber      = 1,
  aboveskip       = 2pt plus 1pt,
  belowskip       = 8pt plus 3pt
}
\lstnewenvironment{code}[0]{}{}

% only shown, not compiled:
\lstnewenvironment{showCode}[0]{\lstset{numbers=none}}{}

% only compiled, not shown:
\newcommand{\hide}[1]{}

% will the real phi please stand up
\renewcommand{\phi}{\varphi}

\newcommand{\gor}{\mathbin{\rotatebox[origin=c]{-90}{$ \geqslant $}}}


% load hyperref as late as possible for compatibility
\usepackage[pdftex]{hyperref}
\hypersetup{
  colorlinks = true,
  pdfborder = {0 0 0},
  breaklinks = true,
  linktoc = all,
  linkcolor = blue,
  urlcolor = magenta
}
\pdfinfoomitdate=1
\pdftrailerid{}
\pdfsuppressptexinfo15


\title{Project Report - Beta}
\author{Group 3 - N. Cohen, T. Coroneo, V. Iyer, G. Kaminer, and L. Madison}
\date{\today}
\hypersetup{pdfauthor={Me}, pdftitle={Project Report - Beta}}

\setlength{\parskip}{10pt plus 1pt minus 1pt}

\DeclareMathOperator{\botbot}{\rotatebox{90}{$\models$}}
\DeclareMathOperator{\toptop}{\raisebox{0pt}[0pt][0pt]{\rotatebox[origin=c]{270}{$\models$}}}


\begin{document}

\maketitle

\begin{abstract}
We implement an explicit model checker for bilateral state-based modal logic
(BSML), as described in \cite{Aloni2024}. BSML has been used to account for
Free-Choice (\textit{FC}) and related inferences which arise from speakers' distaste for
interpretations that verify a sentence by empty configuration  (\textit{neglect-zero tendency}).
\end{abstract}

Free-Choice (\textit{FC}) inferences are instances of conjunctive meaning being unexpectedly
derived from a disjunctive sentence. In the following example, a modalized disjunction yields
a conjunction of modals.

\begin{itemize}
\item You may go to the beach or to the cinema $\rightsquigarrow$ You may go to the beach
 and you may go to the cinema
\item $\Diamond(b\vee c)\rightsquigarrow\Diamond b \wedge \Diamond c$
\end{itemize}

Aloni (2022) explain that in interpreting a sentence, a speaker identifies which are the structures of
reality that would reflect the sentence (the models in which the sentence would be assertable).
In our disjunction, there are four associated worlds:

\begin{enumerate}
\item $W_{bc}$ in which both $b$ and $c$ are true (you go to both the beach and the cinema)
\item $W_b$ in which $b$ is true (you go to the beach)
\item $W_c$ in which $c$ is true (you go to the cinema)
\item $W_z$ in which neither $b$ nor $c$ is true (you don't go anywhere)
\end{enumerate}

We can use BSML to model the information states (i.e. sets of worlds) in which the disjunction $b\vee c$
would be assertable. A proposition is assertable at a state $s$ if it is assertable in all its substates.
A disjunction is assertable in a state $s$ if each of the disjuncts is supported in a substate of $s$.

\begin{enumerate}
\item In a state consisting of $W_b$ and $W_c$, $b\vee c$ is assertable
\item In a state consisting of $W_{bc}$ and $W_b$ (or a state consisting of $W_{bc}$ and $W_c$),
$b\vee c$ is assertable
\item In a state consisting of $W_b$ (or a state consisting of $W_c$), $b\vee c$ is assertable since each of the
disjuncts is supported in a substate. $b$ is supported in $W_b$, and $c$ is supported in the empty state.
\item In a state consisting of $W_z$ and $W_b$ (and any other state which includes $W_z$), $b\vee c$ is
\textit{not} assertable because in $W_z$ both $b$ and $c$ are false, so this state would leave open the
possibility that neither $b$ nor $c$ is the case.
\end{enumerate}

So among the four types of states, $b\vee c$ is assertable in all but the last. This is a problem, because if
$b\vee c$ is assertable In a state consisting of only $W_b$ (or a state consisting of only $W_c$) then we have
that $\Diamond(b\vee c)$ is true while $\Diamond b \wedge \Diamond c$ is false, so the FC inference fails. The
problematic state, then, is the zero-model: one of the states which it uses to satisfy the disjunction is the
empty state.

How do we account for FC inferences then? Aloni argues pragmatically that a speaker would not consider the
zero-model as one of the candidate states. Neglecting the zero model then, the FC inference would hold
because the only states that would support $b\vee c$ would be (1) or (2). To model neglect-zero (to make
sure that $b\vee c$ is not assertable in the zero-model), we require that to satisfy a disjunction, the state
must be the union of two non-empty substates rather than just the union of two substates. So we enrich each
formula with a $[]^+$ function which conjuncts the non-emptiness atom (NE) to each formula and all its
subformulas.

After enrichment, the disjunction $b\vee c$ is no longer assertable in a state consisting of only $W_b$
(or a state consisting of only $W_c$) since the non-emptiness atom (NE) would not be supported in all the substates.
Finally then, if the only states in which the disjunction holds are (1) and (2), then the FC inference holds.

\tableofcontents

\clearpage

% We include one file for each section. The ones containing code should
% be called something.lhs and also mentioned in the .cabal file.


\section{How to use this?}

To generate the PDF, open \texttt{report.tex} in your favorite \LaTeX editor and compile.
Alternatively, you can manually do
\texttt{pdflatex report; bibtex report; pdflatex report; pdflatex report} in a terminal.

You should have stack installed (see \url{https://haskellstack.org/}) and
open a terminal in the same folder.

\begin{itemize}
  \item To compile everything: \verb|stack build|.
  \item To open ghci and play with your code: \verb|stack ghci|
  \item To run the executable from Section \ref{sec:Main}: \verb|stack build && stack exec myprogram|
  \item To run the tests from Section \ref{sec:simpletests}: \verb|stack clean && stack test --coverage|
\end{itemize}


\input{lib/Defs.lhs}

\input{exec/Main.lhs}

\input{test/simpletests.lhs}


\section{Conclusion}\label{sec:Conclusion}


\addcontentsline{toc}{section}{Bibliography}
\bibliographystyle{alpha}
\bibliography{references.bib}

\end{document}
