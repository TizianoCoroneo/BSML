\documentclass[12pt,a4paper]{article}
\usepackage{amsmath}
\input{latexmacros.tex}

\title{My Report}
\author{Me}
\date{\today}
\hypersetup{pdfauthor={Me}, pdftitle={My Report}}

\begin{document}

\maketitle

\begin{abstract}
We implement an explicit model checker for bilateral state-based modal logic
(BSML), as described in \cite{Aloni2024}. BSML has been used to account for
Free-Choice (\textit{FC}) and related inferences which arise from speakers' distaste for
interpretations that verify a sentence by empty configuration  (\textit{neglect-zero tendency}).
\end{abstract}

Free-Choice (\textit{FC}) inferences are instances of conjunctive meaning being unexpectedly
derived from a disjunctive sentence. In the following example, a modalized disjunction yields
a conjunction of modals. 

\begin{enumerate}
\item You may go to the beach or to the cinema $\rightsquigarrow$ You may go to the beach
 and you may go to the cinema
\item $\Diamond(b\vee c)\rightsquigarrow\Diamond b \wedge \Diamond c$
\end{enumerate}


Such an inference is invalid in classical modal logic, since when $b$ is false and $c$ is true $\Diamond(b\vee c)$ is true but
$\Diamond b \wedge \Diamond c$ is false. Yet the inference in FC makes perfect sense in natural language. How
to account for the discrepancy then? Aloni (2022) explain FC inference as a consequence of the \textit{neglect-zero tendency},
the cognitive preference for representations of reality that verify a sentence by concrete occurrences rather
than an empty configuration. For example, though the sentence "I have no red hats" is verified both by a closet
of blue hats and by a closet with no hats, "blue hats" is less cognitively taxing than "no hats", the zero-model.

\tableofcontents

\clearpage

% We include one file for each section. The ones containing code should
% be called something.lhs and also mentioned in the .cabal file.

\input{Howto.tex}

\input{lib/Defs.lhs}

\input{exec/Main.lhs}

\input{test/simpletests.lhs}

\input{Conclusion.tex}

\addcontentsline{toc}{section}{Bibliography}
\bibliographystyle{alpha}
\bibliography{references.bib}

\end{document}
