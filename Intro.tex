\section{Introduction}
This section outlines the motivation for BSML by walking through a simple example from \cite{Aloni2022}. Readers interested only in the implementation can safely skip this section. 

\subsection{Motivating example}
Free-Choice (\textit{FC}) inferences are instances of a disjunctive sentence (``or'') unexpectedly yielding a conjunctive reading (``and''). In the following example, a modalized disjunction (1) yields a conjunction of modals (2).

\begin{enumerate}
\item $\lozenge(b\vee c)$ You may go to the beach \textbf{or} to the cinema
\item $\lozenge b \wedge \lozenge c$ You may go to the beach \textbf{and} you may go to the cinema
\end{enumerate}

Aloni \cite{Aloni2022} posits that speakers interpret a sentence by identifying structures of reality that reflect it. Such a structure, or ``information state'', is a set of possible worlds. 

In our disjunction, there are four associated worlds:

\begin{enumerate}
\item $W_{bc}$ in which both $b$ and $c$ are true (you go to both the beach and the cinema)
\item $W_b$ in which $b$ is true (you go to the beach)
\item $W_c$ in which $c$ is true (you go to the cinema)
\item $W_z$ in which neither $b$ nor $c$ is true (you don't go anywhere)
\end{enumerate}

 We can use BSML to model the information states (i.e. sets of worlds) in which the disjunction $b\vee c$ is assertable (or rejectable). A disjunction $\varphi\vee\psi$ is assertable in a state $s$ if $s$ is the union of two substates $t$ and $u$, where $\varphi$ is assertable in $t$ and $\psi$ is assertable in $u$.

\begin{enumerate}
\item Where state $s_1$ is the union of $W_b$ and $W_c$, $b\vee c$ is assertable since $b$ is assertable in $W_b$ and $c$ is assertable in $W_c$
\item Where state $s_2$ is the union of $W_{bc}$ and $W_b$ (or $W_{bc}$ and $W_c$), $b\vee c$ is assertable since $b$ is assertable in $W_b$ and $c$ is assertable in $W_{bc}$
\item Where state $s_3$ is the union of $W_b$ and the empty set (or $W_c$ and the empty set), $b\vee c$ is assertable since 


since each of the
disjuncts is assertable in a substate. $b$ is assertable in $W_b$, and $c$ is supportable in the empty state.
\item In a state consisting of $W_z$ and $W_b$ (and any other state which includes $W_z$), $b\vee c$ is
\textit{not} assertable because in $W_z$ both $b$ and $c$ are false, so no substate containing $W_z$ would allow assertion of $b$ or $c$.
\end{enumerate}

So, among the four types of states, $b\vee c$ is assertable in all but the last. This is a problem, because if
$b\vee c$ is assertable in a state consisting only of $W_b$ (or a state consisting of only $W_c$) then we have
that $\Diamond(b\vee c)$ is true while $\Diamond b \wedge \Diamond c$ is false, so the FC inference fails. The
problematic state, then, is the zero-model: one of the states which it uses to satisfy the disjunction is the
empty state.

How do we account for FC inferences then? Aloni argues pragmatically that a speaker would not consider the
zero-model as one of the candidate states. Neglecting the zero model then, the FC inference would hold
because the only states that would support $b\vee c$ would be (1) or (2). To model neglect-zero (to make
sure that $b\vee c$ is not assertable in the zero-model), we require that to satisfy a disjunction, the state
must be the union of two non-empty substates rather than just the union of two substates. 
This is modelled by enriching formulas using a \emph{pragmatic enrichment function} which conjuncts to each subformula a non-emptiness atom (\verb|NE|), which requires supporting states to be inhabited.

The enrichment of $b \vee c$ (denoted $[b\vee c]^+$) is no longer assertable in a state consisting of only $W_b$
(or a state consisting of only $W_c$) since \verb|NE| would not be assertable in any substates that could (vacuously) support $c$.
Finally then, since the only states in which the enriched disjunction holds are (1) and (2), the FC inference holds.

\subsection{Our contribution}
-- What is our contribution (model checking + ND)\\
-- Why is model checking/nice important? (Vighnesh blab as in presentation)\\
-- Why is representing ND nice? (Use as interactive theorem prover and possibly automated proof search in future)